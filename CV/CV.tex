\documentclass[a4paper,10pt]{article}
\usepackage[left=1in, right=1in, top=1in, bottom=1in]{geometry}
\usepackage[UTF8, scheme=plain]{ctex}  % 加入 scheme=plain 选项
\usepackage{titlesec, enumitem, hyperref}

\titleformat{\section}{\large\bfseries}{}{0pt}{}  % Formatting section titles

\begin{document}

\begin{center}
    {\LARGE \textbf{张津睿}}\\
    辽宁大连, 邮编: 116021 \\
    邮箱: jerryzhang40@gmail.com \\
    学生邮箱: zhangjr1022@mails.jlu.edu.cn \\
    移动电话: +86-134-7894-0406 \\
    \href{https://github.com/unjerry}{GitHub: unjerry}
\end{center}

\section*{教育经历}
\textbf{吉林大学}, 吉林, 长春  \hfill 预期毕业: 2026  \\
数学学士 \\
相关课程: C语言, 数值分析, 最优控制。

\section*{研究兴趣方向}
Machine Learning, Deep Learning, Reinforcement Learning, Numerical Analysis, PINN \\


\section*{研究经历}
\textbf{学习并开发全部C++实现的ANN}
\hfill 2023年03月 – 2023年04月\\
自学机器学习相关知识,学习了深度学习和强化学习的相关
算法,并自己从零运用学到的数学分析和高等代数知识,运用
C++构建矩阵乘法和自动求导系统实现了一个ANN。
\\
\textbf{学习PINN相关算法并实现burgers方程的PINN训练求解}
\hfill 2023年05月 – 2023年07月\\
在李兹谦学长指导下,学习PINN相关算法并实现burgers
方程的PINN训练求解。后自学了连续介质力学相关方程
实现了对简单流体力学方程的PINN训练求解。并成功
出现了类似卡门涡街的现象。
\\
\textbf{FFT和NTT支持的任意位数高精度整数运算}
\hfill 2023年09月 – 2023年12月\\
使用NTT(数论变换)和FFT(快速傅里叶变换)实现了任意位数高精度整数乘法算法。
\cite{firstApproachGithubProject}.
\\
\textbf{开发计算微分几何的网格细分算法}
\hfill 2024年01月 – 2024年07月\\
在宋海明老师指导下,参与基于C++工程进行
三角形和四边形网格细分算法开发,
并完成第一期项目。
\\
\textbf{arXiv上独立发表论文《Differential Informed Auto-Encoder》}
\hfill 2024年09月 – 2024年12月\\
在数据各阶导数和偏导数构成的像空间中
进行回归分析和机器学习,找到内蕴的微
分方程关系并用神经网络建模。基于PINN
算法实现由神经网络学习到的内蕴关系驱动
的数据再生成。其中微分关系的神经网络构
成编码器,PINN算法构成解码器。由此构成
一个微分信息自编码器。
\cite{zhang2024differentialinformedautoencoder}

\section*{技术技能}
\textbf{编程技能:} 熟练掌握多种编程语言, 包括但不限于, C/C++, Python, MATLAB, LaTeX, CudaC++  \\
其中Python能熟练运营torch和numpy等库进行机器学习和数据分析 \\
C++能熟练运用CMake进行跨平台工程代码构建 \\
\textbf{软件工程:} CMake, git \\
\textbf{数学能力:} 熟练掌握高等代数,能够熟练的用线性空间等理论对机器学习中的问题进行分析建模 \\
熟练掌握基础微积分,并能够进行一定程度的高等数学分析。\\
\textbf{其他:} 有着算法学习经验,能够用C语言实现若干计算机科学领域基础算法
包括但不限于二叉树,二叉查找树,红黑树,树状数组,线段树,动态规划。\\
数据分析,有着实战python爬虫经验,实现从xueqiu.com实时爬取股票数据
并分析。\\
机器学习,独立实现ANN,CNN,RNN,G-CNN等多种神经网络模型。\\
软件工程,熟练运用git进行工程和项目管理、文章撰写等复杂任务。

\section*{获奖经历与荣誉}
\begin{itemize}
    \item 全国大学生数学建模竞赛省级二等奖,2023年 \\
          队员:徐慧妍、朱子璇、张津睿 \\
          负责任务:主要参与部分建模与全部编程任务(编程手)
    \item 美国大学生数学建模竞赛(MCM/ICM)S奖(Successful Participant),2024年 \\
          队员:赵艺婷、晋中宝、张津睿 \\
          负责任务:主要负责模型建立与编程实现
    \item 数学学院积分大赛一等奖,2023年
    \item 担任班级心理委员
\end{itemize}

\section*{Publications and Presentations}
\textbf{Differential Informed Auto-Encoder} | arXiv:2410.18593  \\
% \textbf{Talk Title} | Conference Name, Year

% \section*{Teaching  Leadership}
% \textbf{Teaching Assistant}, Course Name, University, Semester  \\
% \textbf{Math Club President}, Institution, Year

% \section*{Extracurricular Activities}
% - Participated with 
% - Competed in [Hackathon/Competition Name]

\bibliographystyle{plain}  % or another style like unsrt, IEEEtran, etc.
\bibliography{CV}  % references.bib is the file name

\end{document}
